% Options for packages loaded elsewhere
\PassOptionsToPackage{unicode}{hyperref}
\PassOptionsToPackage{hyphens}{url}
%
\documentclass[
]{article}
\usepackage{amsmath,amssymb}
\usepackage{lmodern}
\usepackage{iftex}
\ifPDFTeX
  \usepackage[T1]{fontenc}
  \usepackage[utf8]{inputenc}
  \usepackage{textcomp} % provide euro and other symbols
\else % if luatex or xetex
  \usepackage{unicode-math}
  \defaultfontfeatures{Scale=MatchLowercase}
  \defaultfontfeatures[\rmfamily]{Ligatures=TeX,Scale=1}
\fi
% Use upquote if available, for straight quotes in verbatim environments
\IfFileExists{upquote.sty}{\usepackage{upquote}}{}
\IfFileExists{microtype.sty}{% use microtype if available
  \usepackage[]{microtype}
  \UseMicrotypeSet[protrusion]{basicmath} % disable protrusion for tt fonts
}{}
\makeatletter
\@ifundefined{KOMAClassName}{% if non-KOMA class
  \IfFileExists{parskip.sty}{%
    \usepackage{parskip}
  }{% else
    \setlength{\parindent}{0pt}
    \setlength{\parskip}{6pt plus 2pt minus 1pt}}
}{% if KOMA class
  \KOMAoptions{parskip=half}}
\makeatother
\usepackage{xcolor}
\IfFileExists{xurl.sty}{\usepackage{xurl}}{} % add URL line breaks if available
\IfFileExists{bookmark.sty}{\usepackage{bookmark}}{\usepackage{hyperref}}
\hypersetup{
  hidelinks,
  pdfcreator={LaTeX via pandoc}}
\urlstyle{same} % disable monospaced font for URLs
\usepackage[margin=1in]{geometry}
\usepackage{graphicx}
\makeatletter
\def\maxwidth{\ifdim\Gin@nat@width>\linewidth\linewidth\else\Gin@nat@width\fi}
\def\maxheight{\ifdim\Gin@nat@height>\textheight\textheight\else\Gin@nat@height\fi}
\makeatother
% Scale images if necessary, so that they will not overflow the page
% margins by default, and it is still possible to overwrite the defaults
% using explicit options in \includegraphics[width, height, ...]{}
\setkeys{Gin}{width=\maxwidth,height=\maxheight,keepaspectratio}
% Set default figure placement to htbp
\makeatletter
\def\fps@figure{htbp}
\makeatother
\setlength{\emergencystretch}{3em} % prevent overfull lines
\providecommand{\tightlist}{%
  \setlength{\itemsep}{0pt}\setlength{\parskip}{0pt}}
\setcounter{secnumdepth}{-\maxdimen} % remove section numbering
\ifLuaTeX
  \usepackage{selnolig}  % disable illegal ligatures
\fi

\author{}
\date{\vspace{-2.5em}}

\begin{document}

\hypertarget{superfast-response-theory}{%
\section{Superfast response theory}\label{superfast-response-theory}}

\[
\newcommand{\fvec}{\mathbf{f}}
\newcommand{\evec}{\mathbf{e}}
\newcommand{\avec}{\mathbf{a}}
\newcommand{\amat}{\mathbf{A}}
\newcommand{\umat}{\mathbf{U}}
\newcommand{\kmat}{\mathbf{K}}
\newcommand{\cmat}{\mathbf{C}}
\]

\hypertarget{response-to-edge-mutations}{%
\subsection{Response to edge
mutations}\label{response-to-edge-mutations}}

A pair of forces acting on nodes \(j\) and \(k\) in opposite directions
is resprsented by a vector \[
(\fvec^{jk})^T = (\ldots, -f^{jk}\evec^{jk},\ldots,f^{jk}\evec^{jk},\ldots)
\] Let \[
\avec = \amat \fvec
\] Be a response vector (\(\amat\) may be, for instance, the identity
matrix, \(\cmat^{1/2}\), \(\cmat\), etc.). Then, the response is given
by: \[
a_i^{jk} = (\amat_{ik} - \amat_{ij})\evec_{jk}f_{jk}
\]

Analogously, the response along mode \(n\) is given by: \[
a_n^{jk} = (\amat_{nk} - \amat_{nj})\evec_{jk}f_{jk}
\] where \(A_{ni}\) are elements of the matrix \(\umat^T \amat\), where
\(\umat\) is the matrix of normal-mode vectors \(<i|n>\) .

(Note, if \(\umat\) diagonalises \(\amat\) and \(\alpha\) is the
diagonal matrix of eigenvalues: \(\umat^T\amat = \alpha \umat^T\)).

If \(f_{jk}\) have a distrbution with mean \(0\) and standard deviation
\(\sigma\), then the average response is zero, and the average square
response is: \[
R_i^{jk} = <(a_i^{jk})^2> = \sigma^2 ||(\amat_{ik} - \amat_{ij})\evec_{jk}||^2
\] and the average square response along modes: \[
R_n^{jk} = <(a_n^{jk})^2> = \sigma^2 ||(\amat_{nk} - \amat_{nj})\evec_{jk}||^2
\]

\hypertarget{response-to-site-mutations}{%
\subsection{Response to site
mutations}\label{response-to-site-mutations}}

In the LFENM, mutating a site amounts to mutating the edges of that
site. If the mutations are independent and have the same distribution of
forces, then the response to a site mutation is equal to the sum of
responses over edge mutations: \[
R_i^{j} = \sum_{k \sim j} R_i^{jk} \\
R_n^{j} = \sum_{n \sim j} R_i^{jk}
\] If the distribution of forces is not the same (for instance, in the
LFENM if we use \(\delta l_{jk}\) with the same distributions,
\(f_{jk} = k_{jk}\delta l_{jk}\) will not be identically distributed),
then: \[
R_i^{j} = \sum_{k \sim j} R_i^{jk}<f_{jk}^2> \\
R_n^{j} = \sum_{n \sim j} R_i^{jk}<f_{jk}^2>
\]

\hypertarget{normal-mode-transform}{%
\subsection{Normal-mode transform}\label{normal-mode-transform}}

Note that in the normal-mode resporesentation,
\(\avec \rightarrow \umat^T \avec\).

\hypertarget{superfast-calculation}{%
\subsection{Superfast calculation}\label{superfast-calculation}}

The formulas in this note allow the calculation of average response
without the need to actually simulate mutations and average over them,
thus in principle it should be faster (and more accurate).

The procedure would be:

\begin{verbatim}
1. Calculate a(i, kl) and a(n, kl) for all edges (note: a(n,) = t(U) %*% a(i,))
2. Calculate r(x, kl) = a(x, kl)^2
3. calculate r(x, k) = sum(r(x, kl) * sigma_f(kl)^2)
\end{verbatim}

\end{document}
